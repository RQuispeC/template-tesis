\chapter{Aspectos Generales}
Las generalidades del proyecto de investigación son presentadas en el presente capítulo. A continuación ...

%===============================================================================
\section{Problema de Investigación}

\label{sec:problema}
%--------------------------------------------------------------
\subsection{Descripción del Problema}


La danza forma ...

%--------------------------------------------------------------
\subsection{Formulación del Problema}
Actualmente no se puede ...

%===============================================================================
\section{Antecedentes}

De la bibliografía revisada ...

\begin{table}[H]
\captionof{table}{Métodos presentados en el Cultural Event Recognition Challenge} \label{table:chlearn:methods}
\resizebox{\textwidth}{!}{%
    \begin{tabular}{| c | l |}
    \hline
    \bf Equipo & \bf Método Propuesto\\
    \hline
    UPC-STP & AlexNet, SVM para la clasificación.\\
    \hline
    \end{tabular}}
\footnotesize\it{\textbf{Fuente: } \cite{escalera:2015:cultural-event}}    
\end{table} 


%===============================================================================
\section{Justificación}
\label{sec:justificacion}
Cusco es el principal destino turístico del país: a cada hora del día, 320 turistas pisan suelo cusqueño \footnote{Fuente: diario La República \url{https://goo.gl/Oumg05}}. ...
%===============================================================================
\section{Objetivos}
\label{sec:objetivos}

\subsection{Objetivo General}

obejtivo general

\subsection{Objetivos Específicos}
Dentro de los objetivos específicos se incluyen:
   \begin{itemize}
   \item objetivo 1
   \item objetivo 2
   \end{itemize}
%===============================================================================   
\iffalse
\section{Hipótesis}
\label{sec:hipotesis}

Las técnicas de ...

\subsection{Variables}
\subsubsection{Variables Independientes}
En el procesamiento de imágenes existen variables ...
\begin{itemize}
\item \textbf{variable 1},  Descripción de variable 1
\item \textbf{variable 2},  Descripción de variable 2
\end{itemize}
\subsubsection{Variables Dependientes}
Como variable dependiente ... 
\fi 

\section{Evaluación}

Para la evaluación del desempeño del trabajo se hará uso de:

\begin{equation}
\label{equation:acierto}
Tasa~de~error = \frac{\#~de~predicciones~incorrectas}{\#~de~elementos~de~prueba}
\end{equation}

\begin{equation}
\label{equation:error}
Tasa~de~acierto = 1 - Tasa~de~error
\end{equation}

Se hará uso de \textbf{\emph{Cross Validation}} ...

%===============================================================================
\section{Alcances y Limitaciones}
\label{sec:alcances}
Debido a la complejidad del problema de clasificación de imágenes ...
\begin{itemize}
\item Limitación 1
\item Limitación 2
\end{itemize}

%===============================================================================
\section{Metodología}
Por la naturaleza del problema :
\begin{enumerate}
\item \textbf{Paso 1} Descripción paso 1.
\item \textbf{Paso 2} Descripción paso 2.
\end{enumerate}

%===============================================================================
\section{Contribuciones}
El presente trabajo tiene las siguientes contribuciones:
   \begin{itemize}
   \item Contribución 1
   \item Contribución 2
   \end{itemize}
%===============================================================================

\clearpage
\section{Cronograma de Actividades}
\label{sec:cronograma}
El cronograma de actividades se muestra en la figura \ref{fig:cronograma}. 

\begin{figure}[H]
\caption{Diagrama de Gantt del cronograma de actividades.}
 \scalebox{0.8}{
  \begin{gantt}[xunitlength=1cm,fontsize=\small,titlefontsize=\small,drawledgerline=true]{9}{14}
    \begin{ganttitle}
      \titleelement{2016}{12}
      \titleelement{2017}{2}
    \end{ganttitle}
    \begin{ganttitle}
      \numtitle{1}{1}{12}{1}
      \numtitle{1}{1}{2}{1}
    \end{ganttitle}
    \ganttbar{\textbf{1. Revisión de la Literatura}}{1}{7}
    \ganttbar{\textbf{2. Recolección de imágenes}}{4}{4}
    \ganttbar{\textbf{3. Presentación de plan de tesis}}{8}{1}
    \ganttbar{\textbf{4. Investigación de soluciones}}{8}{3}
    \ganttbar{\textbf{5. Implementación}}{10}{1.5}
    \ganttbar{\textbf{6. Evaluación}}{11}{1}
    \ganttbar{\textbf{7. Presentación de trabajo de tesis.}}{11.5}{1.5}    
  \end{gantt}
  }  
  \label{fig:cronograma}
\end{figure}


